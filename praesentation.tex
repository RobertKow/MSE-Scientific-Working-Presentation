% Beamer-Klasse
\documentclass[aspectratio=43]{beamer} 	% 4:3 Format
%\documentclass[aspectratio=169]{beamer} % 16:9 Format

% Pakete laden
% Paket für Deutsche Sprache (Übersetzungen von Chapter zu Kapitel, 
% richtige Umlaute, richtige % Silbentrennung)
% siehe auch http://de.wikipedia.org/wiki/Babel-System
\usepackage[ngerman]{babel}

% Eingabecodierung, deutsche Umlaute oder die akzentuierten Zeichen sind verfügbar 
% und können direkt eingegeben werden
% siehe auch http://de.wikipedia.org/wiki/UTF8
\usepackage[utf8]{inputenc}

% Ausgabeschriftart von LaTeX festlegen
% siehe auch http://de.wikibooks.org/wiki/LaTeX-Schnellkurs:_Erste_Schritte
\usepackage[T1]{fontenc}

% Standardpfad für Grafiken
\graphicspath{{logos/}{bilder/}}

% Paket zum Erstellen von Plots mit TikZ
\usepackage{pgfplots}
% immer die neueste Version benutzen
\pgfplotsset{compat=newest}
% Verbindung von Linien durch eine schräge Kante
\pgfplotsset{every axis/.append style={line join=bevel}}
% Formatvorlage für Präsentationen
\mode<beamer>{
	\pgfplotsset{
		beamer/.style={
			width=0.8\textwidth,
			height=0.45\textwidth,
			legend style={font=\scriptsize},
			tick label style={font=\footnotesize},
			label style={font=\small},
			max space between ticks=28,
		}
	}
}
\mode<handout>{
	\pgfplotsset{
		beamer/.style={
			width=0.8\textwidth,
			height=0.45\textwidth,
			legend style={font=\scriptsize},
			tick label style={font=\footnotesize},
			label style={font=\small},
			max space between ticks=25,
		}
	}
}
\mode<article>{
	\pgfplotsset{
		beamer/.style={
			width=0.8\textwidth,
			height=0.45\textwidth,
			max space between ticks=35,
		}
	}
}
% neue Größenvorlage für zwei Plots nebeneinander anlegen
\pgfplotsset{
	scriptsize/.style={
		width=0.34\textwidth,
		height=0.1768\textwidth,
		legend style={font=\scriptsize},
		tick label style={font=\scriptsize},
		label style={font=\footnotesize},
		title style={font=\footnotesize},
		every axis title shift=0pt,
		max space between ticks=25,
		every mark/.append style={mark size=7},
		major tick length=0.1cm,
		minor tick length=0.066cm,
	}
}
\pgfplotsset{
	small/.style={
		width=6.5cm,
		height=,
		tick label style={font=\footnotesize},
		label style={font=\small},
		legend style={font=\footnotesize},
		max space between ticks=30,
	}
}
% Legendeneintrage standardmäßig links ausrichten
\pgfplotsset{legend cell align=left}
% Hauptgitternetz zeichnen
\pgfplotsset{xmajorgrids}
\pgfplotsset{ymajorgrids}
% Anzahl der kleinen Teilstriche zwischen zwei großen Teilstrichen
%\pgfplotsset{minor x tick num={3}}
%\pgfplotsset{minor y tick num={3}}
% feines Gitternetz zeichnen
%\pgfplotsset{xminorgrids}
%\pgfplotsset{yminorgrids}
% nur nach den Achsen skalieren
\pgfplotsset{scale only axis}
% Farben wie in MATLAB definieren
\definecolor{matlab1}{rgb}{0,0,1}
\definecolor{matlab2}{rgb}{0,0.5,0}
\definecolor{matlab3}{rgb}{1,0,0}
\definecolor{matlab4}{rgb}{0,0.75,0.75}
\definecolor{matlab5}{rgb}{0.75,0,0.75}
\definecolor{matlab6}{rgb}{0.75,0.75,0}
\definecolor{matlab7}{rgb}{0.25,0.25,0.25}
% Farbreihenfolge wie in MATLAB definieren
\pgfplotscreateplotcyclelist{matlab}{
	{matlab1,solid},
	{matlab2,dashed},
	{matlab3,dashdotted},
	{matlab4,dotted},
	{matlab5,densely dashed},
	{matlab6,densely dashdotted},
	{matlab7,densely dotted}% dies unterdrückt einen Fehler
}
% Farbreihenfolge wie in MATLAB benutzen
\pgfplotsset{cycle list name=matlab}
% Farbreihenfolge von pgfplots benutzen
%\pgfplotsset{cycle list name=color list}
% nur Graustufen benutzen
%\pgfplotsset{cycle list name=linestyles}
% Strichstärke auf 1pt festlegen
\pgfplotsset{every axis plot/.append style={line width=1pt}}
% für deutsche Dokumente ein Komma benutzen
\addto\extrasngerman{\pgfplotsset{/pgf/number format/.cd,set decimal separator={{{,}}}}}
% ein halbes Leerzeichen als Tausendertrennzeichen benutzen
%\pgfplotsset{/pgf/number format/.cd,1000 sep={\,}}
% kein Tausendertrennzeichen verwenden
\pgfplotsset{/pgf/number format/.cd,1000 sep={}}
% Zahlen kleiner als 0.1 auch im fixed-Format ausgeben
\pgfplotsset{/pgf/number format/.cd,std=-2}
% neue Positionen für Legenden anlegen
\pgfplotsset{/pgfplots/legend pos/north/.style={/pgfplots/legend style={at={(0.50,0.97)},anchor=north}}}
\pgfplotsset{/pgfplots/legend pos/south/.style={/pgfplots/legend style={at={(0.50,0.03)},anchor=south}}}
\pgfplotsset{/pgfplots/legend pos/east/.style={/pgfplots/legend style={at={(0.97,0.50)},anchor=east}}}
\pgfplotsset{/pgfplots/legend pos/west/.style={/pgfplots/legend style={at={(0.03,0.50)},anchor=west}}}
\pgfplotsset{/pgfplots/legend pos/outer north/.style={/pgfplots/legend style={at={(0.50,1.03)},anchor=south}}}

% Paket für SI-Einheiten
\usepackage[load-configurations=binary]{siunitx}
% Trennzeichen für Bereiche
\addto\extrasngerman{\sisetup{range-phrase={ bis~}}} 
\addto\extrasenglish{\sisetup{range-phrase={ to~}}}

% Paket, um das Floating in Article-Modus abzuschalten
\usepackage{float}

% Paket, um anderen Zeilenabstand einzustellen, besonders für Tabellen
\usepackage{setspace}

% Paket für ein intelligentes Leerzeichen
\usepackage{xspace}

% Abkürzung für z. B.
\newcommand{\zB}{z.\,B.\xspace}

% Paket für schönere Brüche im Textmodus
\usepackage{xfrac}
% Standardeinstellung mit einem schrägen Bruchstrich
\UseCollection{xfrac}{plainmath}

% schönere Tabellen
\usepackage{booktabs}



% Datenquelle der Logos: http://www.cd.ovgu.de/

%===Design der Universität ===
% Otto-von-Guericke-Universität
\usepackage{style/beamer_ovgu}		% deutsch
%\usepackage{style/beamer_ovgu-en}	% english


%===Fakultäten Design / faculty design===
% Fakultät für Maschinenbau
% faculty of mechanical engineering
%\usepackage{style/beamer_mb} 		% deutsch
%\usepackage{style/beamer_mb-en}	 	% english 

% Fakultät für Verfahrens- und Systemtechnik
% faculty of process and system engineering
%\usepackage{style/beamer_vst}		% deutsch
%\usepackage{style/beamer_vst-en}	% english

% Fakultät für Elektrotechnik und Informationstechnik
% faculty of electrical engineering and information technology
%\usepackage{style/beamer_eit}		% deutsch
%\usepackage{style/beamer_eit-en}	% english

% Fakultät für Informatik
% faculty of computer science
%\usepackage{style/beamer_inf}		% deutsch
%\usepackage{style/beamer_inf-en}	% english

% Fakultät für Mathematik
% faculty of mathematics
%\usepackage{style/beamer_ma}		% deutsch
%\usepackage{style/beamer_ma-en}		% english

% Fakultät für Naturwissenschaften
% faculty of natural science
%\usepackage{style/beamer_nat}		% deutsch
%\usepackage{style/beamer_nat-en}	% english

% Medizinische Fakultät
% faculty of medicine
%\usepackage{style/beamer_med}		% deutsch
%\usepackage{style/beamer_med-en}	% english

% Fakultät für Humanwissenschaften
% faculty of human science
%\usepackage{style/beamer_hw}		% deutsch
%\usepackage{style/beamer_hw-en}		% english

% Fakultät für Wirtschaftswissenschaften
% faculty of economics and management
%\usepackage{style/beamer_ww}		% deutsch
%\usepackage{style/beamer_ww-en}		% english


%===weitere Designs===
% Stil des Lehrstuhls für Elektromagnetische Verträglichkeit laden
%\usepackage{beamer_emv}

% begin der Präsentation / begin presentation
\title[Kurztitel]{Titel der Präsentation}
\author{Max Mustermann}
\institute[Lehrstuhl XYZ]{
	Lehrstuhl für XYZ \\
	Institut für XYZ \\
	Otto-von-Guericke-Universität, Magdeburg
}
\mode<presentation>{\keywords{Schlüsselwörter durch Komma getrennt}}
\date[01.01.2016]%{Datum der Präsentation, \zB 1. Januar 2016}

\begin{document}

\begin{frame}
	\maketitle
	% \maketitle funktioniert auch im Article-Modus
	% \titlepage funktioniert nur bei Präsentationen
\end{frame}

\begin{frame}[label=inhalt]{Gliederung}
	\tableofcontents
\end{frame}

\section{Einleitung}

% nur für die Miniframe-Navigation
\subsection*{}

\begin{frame}
	\frametitle<presentation>{Einleitung}
	\begin{block}{Sinn und Zweck einer Abschlusspräsentation:}
		\begin{itemize}
			\item Darstellung der wichtigsten Ergebnisse der Arbeit
			\item Interesse für die Arbeit/den Bericht zu wecken
			\item Ziel: lehrreich und unterhaltsam zugleich
		\end{itemize}
	\end{block}
	\begin{block}{Vorteile von \LaTeX\ mit der \textsc{beamer}-Klasse:}
		\begin{itemize}
			\item sehr einfach, wenn die Arbeit bereits in \LaTeX\ erstellt wurde
			\item verschiedene Darstellungsvarianten zur Auswahl
			\item Handzettel können einfach mit \emph{beamerarticle} erzeugt werden
		\end{itemize}
	\end{block}
\end{frame}

\section{Hauptteil}

\subsection{Gleichungen}

\begin{frame}
	\frametitle<presentation>{Gleichungen}
	\begin{block}{Satz des Pythagoras:}
		\begin{equation}
			a^2 + b^2 = c^2 \label{eq:pythogoras}
		\end{equation}	\end{block}
	\begin{block}{Es folgt, dass:}
		\begin{align}
			a^2 &= c^2 - b^2 \\
			b^2 &= c^2 - a^2
		\end{align} 	
 	\end{block}
\end{frame}

\subsection{Abbildungen}

\begin{frame}
	\frametitle<presentation>{Abbildungen}
	\begin{figure}[!t]
		\centering
			\includegraphics[height=5cm]{bilder/unilogo}
		\caption{Altes Logo der Universität}
		\label{fig:unilogo}
	\end{figure}
\end{frame}

\begin{frame}{Diagramme}
	\begin{figure}[!t]
		\centering
			\begin{tikzpicture}
				\begin{axis}[
					beamer,
					xlabel={Zeit, $t$ (in \si{\milli\second})},
					ylabel={Spannung, $u(t)$ (in \si{\volt})},
					xmin=0,xmax=20,
					ymin=-350,ymax=350,
					legend pos=south west,
					% Beispiel für das Einbinden einer Linie durch Angabe einer Formel
					]
					\addplot+[
						domain=0:20,
						samples=101,
					] {sin(deg(x*2*pi/20))*sqrt(2)*230};
					\addlegendentry{sinusförmig};
					\addplot+[
						domain=0:20,
						samples=101,
					] {cos(deg(x*2*pi/20))*sqrt(2)*230};
					\addlegendentry{cosinusförmig};
				\end{axis}
			\end{tikzpicture}
		\caption{Harmonischer Zeitverlauf einer Spannung mit einer Frequenz von \SI{50}{\hertz} und einem Effektivwert von \SI{230}{\volt}}
		\label{fig:u_t_sinus}
	\end{figure}
\end{frame}

\subsection{Tabellen}

\begin{frame}
	\frametitle<presentation>{Tabellen}
	\begin{table}
		\caption{Beispieltabelle}
		\label{tab:beispiel}
		\centering
			\begin{tabular}{cc}
				\toprule
				Variable & Bedeutung \\
				\midrule
				$t$ & Zeit \\
				$U$ & Spannung \\
				\bottomrule
			\end{tabular}
	\end{table}
\end{frame}

\section{Zusammenfassung}

% nur für die Miniframe-Navigation
\subsection*{}

\begin{frame}
	\frametitle<presentation>{Zusammenfassung}
	\begin{block}{Ergebnisse:}
	  \begin{itemize}
	  	\item Kurzfassung der wichtigsten Ergebnisse
	  	\item Aufzeigen nicht gelöster Probleme und offener Fragen
	  	\item eventuell Ausblick auf weitere geplante Forschungsarbeiten
		\end{itemize}
	\end{block}
	\begin{block}{Fragen der Zuhörer:}
	 \begin{itemize}
	 	\item genügend Zeit für Fragen einplanen
	 	\item optional: zusätzliche Folien für wahrscheinliche Fragen
	 \end{itemize}
	\end{block}	
\end{frame}

% nur für die Miniframe-Navigation
\section*{}

\begin{frame}<beamer>{}
	\begin{center}
		Vielen Dank für Ihre Aufmerksamkeit!
	\end{center}
	\begin{center}
		Gibt es Fragen?
	\end{center}
\end{frame}

\end{document}